\documentstyle[11pt,html]{article}
\special{psfile=logo.ps hoffset=-65 voffset=-60}
\renewcommand{\textfraction}{0.0}
\renewcommand{\topfraction}{1.0}
\renewcommand{\bottomfraction}{1.0}
%
%  For testing features in latex2html without taking forever.
%
% last updated 05dec95
%
% fonts
% {\sc SMALL CAPS} - computer programs names
% {\sl emphatic}   - file names
% {\tt typewriter} - {\sc SCHED}, setup file parms
%

\def\rcwbox#1#2#3#4#5{
\begin{description}
\item[Argument:] #1
\item[Options:]  #2
\item[Default:]  #3
\item[Usage:]    #4
\item[Example:]  #5
\end{description}
}

\begin{document}
\title{LATEX2HTML TESTER}
\author{R.C. Walker}
\date{1995 December 5}
\maketitle

Here is some text to go at the start of the table of contents.
Ver 2

\tableofcontents
%\listoftables
%\listoffigures

And this comes after.

\newpage
\section{INTRODUCTION}
\label{SEC:INTRO}

Program {\sc SCHED} was originally written in the late 1970's at
Caltech as the scheduling program for VLBI projects recorded in
Mark~II format. It can provide the necessary control files for
essentially all telescopes that do, or used to do, Mark~II
observations. There should be enough information in
Sections~\ref{SEC:OVER} and {\tt SEC:MKII} to allow scheduling of
simple Mark~II VLBI observations. However, currently {\sc SCHED} is
mainly used for scheduling VLBI projects in VLBA and Mark~III formats
on the VLBA and VLA. It can also provide control files for other
antennas that use the VLBA style files to control their tape
drives. In addition, {\sc SCHED} can be used to schedule many types of
VLA observations besides just VLBI ones. When run, {\sc SCHED}
produces operator schedules and telescope control files for each
telescope. {\sc SCHED} also produces a summary file giving an overview
of the schedule. To schedule global Mark~III observations, program
{\sc PC-SCHED} (from Haystack) should be used to create the so-called
drg file and the Mark~III station snap files. Very stripped down
inputs for this {\sc SCHED} can be read into {\tt PC-SCHED} if
desired.

Recent versions of {\sc SCHED} were compiled and tested under the VMS
(VAX), UNIX (CONVEX and SUN), and MS-DOS (PC) operating systems.
Currently, the program is only being developed and tested under UNIX
(SUN), but the system files for the other operating systems are still
available and the programming style has not changed. It should still
work under VMS, at least. Under an old MS-DOS, {\tt SCHED} simply got
too big so it may not work there any more.

\section{OVERVIEW}
\label{SEC:OVER}

Program {\sc SCHED} can be used to generate:
\begin{enumerate}
\item Operator schedule files and computer control files for most 
observatories for Mark~II VLBI observations. This is the primary use
of {\sc SCHED} and determines most defaults. Cover information, such
as the PI's name and address, are included in these files.

\item Observe files for some types of VLA-only observations. {\sc 
SCHED} can produce source cards along with limited varieties of
``//LO'', ``//FI'', and ``//PM'' cards.

\item VLBA control files for Mark~III VLBI observations.

\item Observe files and VLBA recorder control files for Mark~III VLBI 
observations at the VLA.

\item VLBA control files for single dish pointing and other testing.
\end{enumerate}

{\sc SCHED} is currently (1992) the primary scheduling program
worldwide for Mark~II VLBI and for all types of VLBA observations.
{\sc SCHED} is normally run from a input control file, called a {\tt
SCHED} keyin file hereafter, which contains commands in the 
 common to all Caltech 
VLBI package programs. The {\sc SCHED}
keyin file is generated using any text editor. Users unfamiliar with
the Caltech package should read Section {\tt SEC:KEYIN}. Within the
{\sc SCHED} keyin file, a series of observations are divided into
scans. A separate group of input keyin parameters, ending with a
``/'', is used to specify a scan. Most parameters default to their
previous values so, generally, only items that change need be
repeated. {\sc SCHED} also writes cover letter information in the
schedule files and control files via values for various keyin
parameters; this cover letter information includes the PI's name,
address, phone number, project type (e.g., Mark~III mode~B at 4cm),
plus other details.

{\sc SCHED} requires the following ancillary information via either
external files or their in-line versions:
\begin{description}
\item [Source and station catalog files:] {\sc SCHED} uses two 
catalog files, one for sources and one for stations. These catalogs
are also in keyin format. Details of the catalog formats and usage are
given in Section {\tt SEC:CAT}.

\item [Setup file:] For any VLBA telescope control file, and for 
control of the VLBA tape system at some non-VLBA stations, {\sc SCHED}
requires extensive information on the details of the hardware
setup. These details are typically the same for all observations of
the same basic type (e.g., Mark~III mode~A at 6cm). {\sc SCHED}
obtains such information from setup files. The VLBA staff maintains a
standard set of such setup files at the Array Operations Center. For
Mark~II programs, {\sc SCHED} can be run without a setup file
specification to check that the schedule is as intended. Then the {\sc
SCHED} keyin file can be sent to the AOC (via, for example, the normal
aspen schedule distribution system) where the appropriate setup file
specification will be added and the final schedule files and control
files will be produced. Setup files are in keyin format.

\item [Tape initialization file:] {\sc SCHED} reads this to obtain 
startup configuration information for the tapes at the sites. This is
mainly for VLBA observations which may put more than one project on a
tape. It is expected that tape initialization files will only be used
by the VLBA operations staff. Users can ignore them in schedule
planning. As with the setup files, VLBA operations must have the {\sc
SCHED} keyin file if a tape initialization file will be needed to
produce the final control files. Tape initialization files are also in
keyin format.

\item [Line initialization file:] Doppler calculations can be 
done for spectral line sources based on velocities in the source
catalog file and rest frequencies in a separate line initialization
file (keyin format, of course). This file is not needed for continuum
projects
\end{description}

The output of {\sc SCHED} consists of three types of files:
\begin{description}
\item [Operator schedule files:] These are intended to be read by 
personnel at the observatories, although they are also useful for the
scheduler. They contain the important parameters of each scan, along
with information about the pointing position (azimuth, elevation, hour
angle, parallactic angle and slew distances). Cover information is
also included. Warnings appear for sources that are too low or within
10 degrees of the Sun. These files have names like {\sl bq001sch.xx},
where {\tt bq001} is an example project code specified with keyin
parameter {\tt EXPCODE} and {\sl xx} is the station code from the
station catalog.

\item [Telescope control files:] These files are intended to be
used to control the telescope computers. The type given for each
station is controlled by a parameter in the station catalog. These
files have names like {\sl bq001crd.xx}, where {\tt bq001} is an
example project code specified with keyin parameter {\tt EXPCODE} and
{\sl xx} is the station code from the station catalog. For non-VLBA
stations with VLBA data aquisition systems controlled by the VLBA
software (VLA, Green Bank and Bonn), two files can be provided. The
{\sl bq001crd.xx} file is a VLBA format file to control the data
aquisition system including tape drive, while the {\sl bq001obs.xx}
file is the local computer control file for use by the telescope
on-line system. All of these files contain cover information.

\item [Project summary file:] This file provides an overview of the 
project including the elevation of the source at each station and the
times of tape changes. These file have names like {\sl bq001.sum},
where {\tt bq001} is an example project code specified with keyin
parameter {\tt EXPCODE}.
\end{description}

Users of older versions of {\sc SCHED} should read
Section{\tt SEC:CHANG} to avoid problems.

{\sc SCHED} is not the same program as {\sc PC-SCHED}. That program,
which was written at Haystack, is the primary program for scheduling
of Mark~III observations. The VLBA staff (or any user) can convert the
{\sc DRUDG} file from {\sc PC-SCHED} (or {\sc SKED}) into a {\tt
SCHED} keyin file using {\sc SKEDCONV}. This is the standard way in
which VLBA and VLA control files are produced for Mark~III programs.

A note on notation: the names of computer programs will be given in
{\sc SMALL CAPS} font, the names of files will appear in {\sl slant}
font, and the names of {\sc SCHED} and setup file parameters will
appear in {\tt typewriter} font. Some parameters may be abbreviated;
where both compulsory and optional characters are displayed, the
compulsory ones will be shown in {\tt UPPER-CASE TYPEWRITER} font and
the optional ones shown in {\tt lower-case typewriter} font.

\end{document}

